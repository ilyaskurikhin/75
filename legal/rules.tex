\documentclass[12pt,a4paper]{article}
\usepackage[utf8]{inputenc}
\usepackage[english]{babel}
\usepackage{amsmath}
\usepackage{amsfonts}
\usepackage{amssymb}

\author{Ilya Skurikhin}
\title{75 Karma Rules}

\begin{document}

\section{Definitions}

\paragraph{Apartment}

The living community ("colocation") situated at 75, Rue de Genève, 1004 Lausanne Switzerland.

\paragraph{75er}

A person who is living in the Apartment, and who's lease is substantiated by a current and valid lease agreement ("bail à loyer").

\paragraph{We}

(us, our) are the developers of the Karma Platform. 

\paragraph{You}

(your) are the user of the Karma Platform ("KP"); delegates will be considered users if exercising a power of attorney on the behalf of a 75er.
	
\paragraph{Karma}

are points that can be received and transferred by You. The official abbreviation of the Karma is "K".

\paragraph{Karma Platform}

is the web interface through which users may make transfers of points and consult their balance. 

\section{Karma Balance}

\begin{enumerate}
	\item When a 75er begins their lease, they can receive one Karma Balance.
	\item All new Karma Balances will contain 100 K.

\end{enumerate}

\section{Sending Karma}

\begin{enumerate}
	\item You may transfer all or part of the Karma in your Karma Balance to the Karma Balance of another 75er. 
	\item Transactions may mention a comment describing the motive of the transaction.
	\item Transactions are made public by Us on the Karma Platform. 
	\item The transaction comment is public by default, but can be recorded privately, therefore making only the recipient, expediter and amount public.
\end{enumerate}

\section{Claiming Karma}

\begin{enumerate}
	\item 
\end{enumerate}

\end{document}