\documentclass[12pt,a4paper]{article}
\usepackage[latin1]{inputenc}
\usepackage[english]{babel}
\usepackage{amsmath}
\usepackage{amsfonts}
\usepackage{amssymb}
\usepackage[left=2cm,right=2cm,top=2cm,bottom=2cm]{geometry}
\usepackage{fancyhdr}

\pagestyle{fancy}
\fancyhf{}
\lfoot{v. 2017-02-09}
\cfoot{Page \thepage}
\rfoot{}
\renewcommand{\headrulewidth}{0pt}

\setlength\parindent{0pt}

\author{Ilya Skurikhin}
\title{Rules of Karma}


\begin{document}

\begin{center}
{\Huge Rules of Karma}
\end{center}

\vspace{2cm}

These are the first Rules of Karma. They will be valid for the first month of activity of the Karma Platform. We will most likely change them thereafter.

\section{Karma Balance}

\begin{enumerate}
	\item When a Resident begins their lease, they can receive one Karma Balance.
	\item All new Karma Balances will contain 1000 K.
	\item Karma Balances must remain positive and current.
\end{enumerate}

\section{Sending Karma}

\begin{enumerate}
	\item You may transfer all or part of the Karma in your Karma Balance to the Karma Balance of any Resident. 
	\item Transactions may mention a comment describing the motive of the transaction.
	\item Transactions are made public on the Karma Platform. 
	\item The transaction comment is public by default, but can be recorded privately, therefore making only the recipient, expediter and amount public.
\end{enumerate}

\section{Claiming Karma}

\begin{enumerate}
	\item You may claim back Karma from a Resident, up to the amount that you have given them in the current calendar month.
\end{enumerate}

\end{document}